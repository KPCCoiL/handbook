%\section{はじめに}

理学部物理学科3年の講義「計算機実験I」および「計算機実験II」は、理論・実験を問わず、学部〜大学院〜で必要とされる現代的かつ普遍的な計算機の素養を身につけることを目標としている。
基礎的な数値計算アルゴリズムとその応用を学習するだけでなく、実習を通じて、計算機の操作、C言語を用いたプログラミング技術、\LaTeX による科学論文の作成技術などを学ぶ。

本冊子は、「計算機実験I」および「計算機実験II」の時間に、より効果的に実習を進めることができるよう、必要とされる技術的な点をまとめたものである。
大いに役立てて欲しい。
また、間違いや不明な点などがある場合には、担当教員、あるいはティーチングアシスタントまでぜひ知らせてほしい。

実習は主として教育用計算機システムECCSの端末(iMac)を用いるが、iMacへのログイン方法や基本的な操作方法、Webやメールの利用方法などについては、本書では触れていない。
教育用計算機システムの「利用の手引」(http://www.ecc.u-tokyo.ac.jp/guide/current/)に詳しい解説があるので、そちらを参照のこと。
